\chapter{Conclusion and Future Work}
\section{Conclusion}
This documented presented an infrastructure dedicated to MPI Performance Engineering, enabling introspection of MPI runtimes.
To serve that purpose, this infrastructure utilized the MPI Tools Information Interface, introduced in the MPI 3.0 standard.
\par With TAU, we gathered existing components --- namely the TAU Performance System and MVAPICH2 and extended them to fully exploit features offered by MPI\_T. We demonstrated different usage scenarios based on specific sets of MPI\_T Performance and Control Variables exported by MVAPICH2. The results produced by our experiments on a combination of synthetic and production applications validate our approach and open broad perspectives for future research.
\par With Caliper, I presented an infrastructure that enables performance introspection through MPI\_T. As compared to TAU, I chose to experiment with a different strategy to sample PVARs. I demonstrated how this strategy could lead to a more meaningful way to analyze PVARs, even if the overheads associated with this strategy were high.
\section{Future Work}
We plan to enrich our infrastructure by exploring the following areas of research:
\begin{itemize}
	\item Develop an infrastructure to express autotuning policies in a more generic fashion
        \item Enrich MPI\_T support in MVAPICH2 to enable introspection and tuning for a wide range of applications and communication patterns
	\item Study the challenges in providing an interactive performance engineering functionality for end users
\end{itemize}

