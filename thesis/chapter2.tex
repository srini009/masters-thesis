\chapter{MPI Tools Information Interface}
In order to address a lack of a standard mechanism to gain insights into, and to manipulate the internal behavior of MPI implementations, the MPI Forum introduced the MPI Tools Information Interface (MPI\_T) in the MPI 3.0 standard~\cite{MPI_3_1}. The MPI\_T interface provides a simple mechanism that allows MPI implementers to expose variables that represent a property, setting or performance measurement from within the implementation for use by tools, tuning frameworks, and other support libraries. The interface broadly defines access semantics of two variable types: \textit{control} and \textit{performance}. The former defines semantics to list, query and set control variables exposed by the underlying implementation. The latter defines semantics to gain insights into the state of MPI using counters, timing data, resource utilization data, and so on. Rich metadata information can be added to both kinds of variables. \par
\textit{Control variables (CVARs)} are properties and configuration settings that are used to modify the behavior of the MPI implementation. A common example of such a control variable is the \emph{Eager Limit} - the upper limit until which messages are sent using the Eager protocol. An MPI implementation may choose to export many environment variables as control variables through the MPI\_T interface. Depending on what the variable represents, it may be set once before \verb+MPI_Init+ or may be changed dynamically at runtime. Further, the interface allows each process freedom to set its own value for the control variable provided the MPI implementation supports it. The MPI\_T interface provides API's to read, write and query information about control variables and external tools can use these API's to discover information about the control variables supported. \par
\textit{Performance variables (PVARs)} can represent internal counters and metrics that can be read, collected and analyzed by an external tool. An example of one such PVAR exported by MVAPICH2 is \verb+mv2_vbuf_total_memory+ which represents the total amount of memory used for internal communication buffers within the library. In a manner similar to CVARs, the interface specifies API's to query and access PVARs. MPI\_T interface allows multiple in-flight performance sessions so it is possible for different tools to \emph{plug into} MPI through this interface. \par
The MPI\_T interface allows an MPI implementation to export any number of PVARs and CVARs, and it is the responsibility of the tool to discover these through appropriate API calls, and use them correctly. There are no fixed events or variables that MPI implementations must support - complete freedom is granted to the implementation in this regard.
