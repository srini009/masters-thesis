\chapter{Discussion}

MPI\_T allows a performance profiler such as TAU or Caliper to play a more active role in MPI performance engineering. As I have demonstrated with experiments with TAU on AmberMD, SNAP, and 3DStencil, there can be significant memory savings in tracking and freeing unused virtual buffers inside MVAPICH2. Such opportunities for fine-tuning MPI library behavior would not have been possible without a close interaction between this two software.
\section{Design Differences Between TAU and Caliper}
\par Although both TAU and Caliper offer MPI performance introspection capabilities through MPI\_T, they differ significantly in the design and implementation of this support. TAU primarily relies on an interrupt-based mechanism to sample the MPI\_T interface, while Caliper relies on an \textit{event} to trigger the MPI\_T sampling routine. This has far-reaching consequences to the overheads involved in introspecting the MPI\_T interface. 
\par TAU's interrupt-based mechanism is extremely light-weight --- it adds almost no noticeable overhead to application runtime even when sampling at the rate of once every second. The event-based scheme implemented in Caliper is expensive --- the overheads generated by both the snapshot triggering mechanisms discussed in this document are prohibitively high. 
\par Although TAU has support for context events, PVARs are stored as user events (they are treated as regular counters) --- as a result, it is not possible in the current design to add metadata information when sampling PVARs. Caliper, on the other hand, has a more flexible API --- the user can define attributes with a set of properties. Specifically, the snapshotting mechanism in Caliper offers a convenient way to add rich context to PVAR data that is collected. This enables a more \textit{meaningful} analysis of PVAR values at the end of a profiling run --- a user can attribute PVAR contributions to specific code sections.
\par TAU clearly has a broader level of support for MPI\_T--- it supports performance monitoring, autotuning, and recommendation generation through MPI\_T in addition to performance introspection. The plugin design in TAU specifically enables support for a broader range of MPI implementations. Caliper at the moment does not support any of these additional features. 
\par The lack of a GUI for performance analysis in Caliper makes in particularly hard to analyze PVARs effectively. Moreover, the profile and trace files are written in a custom format --- the use of well-established tools to view these files is therefore limited. However, it must be noted that Caliper has support for a tool API that enables other tools such as TAU to "plug-in" to Caliper at runtime to extract the collected performance data. This support has not been explored in this work.
\section{A Note on the MPI\_T Interface Specification}
\par Both these tools rely on a PMPI wrapper to generate MPI profiling information. In addition to performing this task, the wrapper in Caliper allocates handles for PVARs that are bound to MPI objects. Although not currently implemented, it is certainly feasible to implement such a functionality within TAU as well. This feature of the MPI\_T specification is particularly interesting as it enables a more fine-grained performance analysis of MPI. However, support for this feature is limited --- OpenMPI is the only implementation that supports this feature. 
\par The MPI\_T standard specification allows an MPI library to dynamically export additional PVARs as and when they become available (through dynamic loading of shared objects). In my opinion, this feature can be supported by tools only by incurring a significant cost in terms of implementation complexity and performance degradation. The solution that would ensue would invariably be ugly by design. As it involves memory allocation, a tool must be careful about when it allocates handles for the additional PVARs (CVARs). Based on the experience with TAU, this certainly cannot be done inside an interrupt handler. Restricting MPI implementations in a way that ensures that they export \textit{all} PVARs (CVARs) during \verb+MPI_Init+ can significantly reduce this complexity. 
\section{Bridge}
In this chapter, we have described how TAU and Caliper differ in their approach to implementing MPI\_T based performance introspection. We have also discussed how these design choices affect runtime overheads. In the concluding chapter, we will describe current efforts in advancing the MPI\_T support in TAU and touch upon future directions for research.
