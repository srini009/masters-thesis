\abstract{The desire for high performance on scalable parallel systems is increasing the complexity and tunability of MPI implementations. The MPI Tools Information Interface (MPI\_T) introduced as part of the MPI 3.0 standard provides an opportunity for performance tools and external software to introspect and understand MPI runtime behavior at a deeper level to detect scalability issues. The interface also provides a mechanism to fine-tune the performance of the MPI library dynamically at runtime.
\par This thesis describes the motivation, design, and challenges involved in developing an MPI performance engineering infrastructure using MPI\_T for two performance toolkits --- the TAU performance system, and CALIPER. 
I validate the design of the infrastructure for TAU by developing optimizations for production and synthetic applications. I show that the design of the MPI\_T runtime introspection mechanism in CALIPER plays a role in determining the sampling overheads.
\par This thesis includes previously published co-authored material.}
