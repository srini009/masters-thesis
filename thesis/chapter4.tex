\chapter{Design of MPI\_T support in CALIPER}

Caliper is an application introspection tool that relies on source code annotations to collect information and perform profiling related tasks. I shall first provide a basic overview of relevant Caliper concepts before describing the MPI\_T support in Caliper. 

\section{Caliper Concepts}
\subsection{Caliper API}
Caliper provides an application level API that acts as the portal for carrying out performance measurements. Caliper also provides high-level annotation macros that are user-friendly. The basic idea behind the source-code annotation API is to associate performance measurements with user-defined, high-level \textit{context information}. These source code annotations act as hooks for background processing. Caliper is built into a library and linked into the application. Figure \ref{fig:caliexample}\footnote{Image taken from: https://llnl.github.io/Caliper} is an example of a Caliper-annotated \verb+C+\texttt{++} source code.
\subsection{Attributes: Caliper's Building Blocks}
Caliper provides a generic \textit{key-value} data model for storing performance data of all kinds. Caliper \textit{attributes} are the basic elements of the Caliper data model. The keys need to have a unique name and a type. They can also optionally have properties which determine how the attributes get processed. An example would be an attribute to track PAPI counters or an attribute to track the total time spent inside a routine or code section. 
\par Among all the properties that an attribute can have, the most important attribute in the context of MPI\_T is the \verb+AS_VALUE+ property. Attributes with the \verb+AS_VALUE+ property set to true cannot be nested. For example, the attribute to track PAPI counters cannot be nested, but the attribute to track the time spent inside a routine is nested.
\begin{center}
	\begin{figure*}[tbp!]
         \centering
		\includegraphics[scale=0.3, width=\columnwidth, keepaspectratio]{figures/cali-example}
		\caption{Caliper~\cite{CALIPER} Annotated Source Code}
		\label{fig:caliexample}
	\end{figure*}
\end{center}

\subsection{Blackboards and Snapshots}
Whenever a performance measurement is made by use of Caliper's measurement API, the values of one or more attributes are updated in an internal data-structure referred to as the \textit{blackboard}. This blackboard is a runtime buffer that is used to combine active attributes, and is updated by Caliper data providers (annotations).
\par A \textit{snapshot} saves the current context of the blackboard. A snapshot can be triggered independently of blackboard updates. Additional information can be added to the snapshot via callbacks to snapshot events. 
\subsection{Services}
Caliper \emph{services} are the basic building blocks that can be combined freely to realize advanced profiling/tracing capabilities. Services are essentially plugins that register callbacks for events of interest. During Caliper initialization, the registered initialization function of each required service is invoked, and the service then performs start-up related tasks inside this initialization function. 
\par An example of a service is the \textit{MPI} service. The MPI service keeps track of the time spent inside MPI calls by utilizing the PMPI interface. The \textit{recorder} service writes Caliper snapshot records into a file using a custom text-based I/O format. The recorder service in conjunction with the MPI service can be used to gather a basic profile of an MPI application. Figure \ref{fig:caliservices} is an illustration of this use case. 
\begin{center}
	\begin{figure*}[bp!]
         \centering
		\includegraphics[scale=0.7, keepaspectratio]{figures/cali-services}
		\caption{MPI Profiling: Caliper~\cite{CALIPER} Service Flow}
		\label{fig:caliservices}
	\end{figure*}
\end{center}

\section {MPI\_T Service}
This section describes the design of the MPI\_T service that performs runtime MPI library introspection through the MPI\_T interface. As of the time being, this service does not support performance monitoring or tuning through the MPI\_T interface. 
\section{Service Registration}
During the registration phase for the MPI\_T service, the MPI\_T performance session is created. The handles for all the PVARs exported at the time of service registration are allocated. It is important to note that service registration may happen before \verb+MPI_Init+ is invoked. If this is the case, the number of PVARs exported would be zero. The design should account for this scenario.
\par In the next section, I shall discuss the complexities that arise with allocating handles for PVARs in detail, and how the design addresses these issues.
\section{PVAR Handle Allocation}
Before a tool can read the value of a PVAR, it must first allocate a \emph{handle} for the PVAR. The MPI\_T interface specifies a function that allows a tool to know the number of PVARs exported by an MPI implementation at any given point in time. Recall that the number of PVARs exported can dynamically increase, and that PVARs can be bound to MPI objects. Like TAU, Caliper only supports PVARs that are exported immediately after \verb+MPI_Init+ by invoking the handle allocation routine inside the PMPI wrapper for \verb+MPI_Init+. Any increase in the number of PVARs after this point is ignored by Caliper. 
\par PVARs can be bound to MPI objects, and such PVARs can provide fine-grained detail about MPI. For example, a PVAR representing the number of messages sent can potentially be bound to an MPI communicator object. This way, it would be possible for a tool to distinguish the quantity of communication across communicators/process groups, instead of presenting an aggregated view to the user.
\par Handles for PVARs not bound to any object (\verb+MPI_T_BIND_NO_OBJECT+) can be allocated at any time --- specifically, this is done inside the registration phase for the MPI\_T service, and inside the PMPI wrapper for \verb+MPI_Init+. In order to allocate handles for PVARs bound to MPI objects, we need a reference (address) for the MPI object in question. The ideal location for the MPI\_T service to grab these references would be during MPI object creation. Briefly, the following steps are necessary to allocate such handles:
\begin{itemize}
\item Identify the corresponding MPI object creation routine for the object in question
\item Intercept the object creation routine (through PMPI)
\item Allocate handles for all PVARs bound to the given object type
\end{itemize}
It is possible that multiple handles are associated with PVARs bound to MPI objects. The supported MPI object types in MPI\_T and their corresponding object routines used to allocate handles are presented in Table \ref{tab:caliperhandles}.
\begin{table*}[!htbp]
  \centering
  \small
  \captionsetup{justification=centering}
  \caption{Caliper: PVAR Handle allocation routines for supported MPI object types}
  \label{tab:caliperhandles}
  \resizebox{0.65\columnwidth}{!}{\begin{tabular}{|c|l|}
    \toprule
    MPI Object Type&MPI Object Creation Routine\\
     \midrule
     MPI Communicator&MPI\_Comm\_Create\\
     MPI Error Handler&MPI\_Err\_handler\\
     MPI File&MPI\_File\_open\\
     MPI Groups&MPI\_Group\_create\\
     MPI Reduction Operators&MPI\_Op\_create\\
     MPI Info Objects&MPI\_Info\_create\\
     MPI Window Objects&MPI\_Win\_create\\
     MPI Datatypes&\textit{Not Supported}\\
     MPI Message Objects&\textit{Not Supported}\\
     MPI Request Objects&\textit{Not Supported}\\
  \bottomrule
\end{tabular}}
\end{table*}

\section{PVAR Classes and Notion of Aggregability}
Depending on what they represent, PVARs are categorized by the MPI standard into counters, state variables, watermarks, etc., and are handled differently by Caliper. We define the notion of \textit{aggregatability} as follows: Any PVAR on which it is \emph{meaningful} to apply one or more of the operators --- SUM, MAX, MIN, AVG, COUNT is defined as aggregatable.
\par Along with other information, a call to \verb+MPIT_pvar_get_info+ returns the \emph{CLASS} to which the PVAR belongs. Below we describe the various PVAR classes supported by the MPI standard and how each class is handled by Caliper:
\begin{itemize}
	\item \verb+MPI_T_PVAR_CLASS_TIMER+, \verb+MPI_T_PVAR_CLASS_AGGREGATE+, \verb+MPI_T_PVAR_CLASS_COUNTERS+: These are free-counting, monotonically increasing values. As such, they are not aggregatable, but by storing the previous value for these counters and timers, the difference between the current and previous value is a derived metric that is aggregatable by use of \emph{SUM}, \emph{MAX}, \emph{MIN}, \emph{AVG} operators. Storing this difference is more useful than just the raw counter values, as one would typically be interested in the \emph{change} caused to any of these PVARs rather than the raw value itself.
	\item \verb+MPI_T_PVAR_CLASS_STATE+: Represents MPI state at any instant in time. Non-aggregatable value.
	\item \verb+MPI_T_PVAR_CLASS_SIZE+: Represents size of an MPI resource. Non-aggregatable value.
	\item \verb+MPI_T_PVAR_CLASS_LEVEL+, \verb+MPI_T_PVAR_CLASS_PERCENTAGE+: Represents the instantaneous level or percentage utilization of an MPI resource. It is meaningful to apply the \emph{AVG}, \emph{MIN}, \emph{MAX} operators, and hence these classes are aggregatable.
	\item \verb+MPI_T_PVAR_CLASS_HIGHWATERMARK+, \verb+MPI_T_PVAR_CLASS_LOWWATERMARK+: As such, these classes are non-aggregatable. However, one can define aggregatable derived metrics out of these PVARs. Specifically, Caliper defines two derived metrics --- A boolean that tells us if the watermark has gone up from the last time it was read, and a double value specifying the \emph{change} in the value between successive reads. Both of these derived metrics are aggregatable quantities as one can apply the \emph{COUNT} and/or \emph{SUM} operator to them.
	\item \verb+MPI_T_PVAR_CLASS_GENERIC+: PVARs that do not fall into any of the above classes. These PVARs would need to be handled on a case-by-case basis, and thus, for now, we define these to be non-aggregatable.
\end{itemize}

\section{Creating Caliper Attributes for PVARs}
The basic data unit in Caliper is an attribute. An attribute is a key-value pair that has certain properties. For each PVAR exposed by the MPI library, Caliper defines an attribute with the same name as the PVAR. Each PVAR attribute has the following properties:
\begin{itemize}
	\item \verb+CALI_ATTR_AS_VALUE+ - We do not want "stacking" semantics for PVAR values. They should be treated much the same way as PAPI counters.
	\item \verb+CALI_ATTR_SCOPE_PROCESS+ - PVARs are defined on a per-rank basis
	\item \verb+CALI_ATTR_SKIP_EVENTS+ - We do not want callbacks to be triggered every time the attribute for a PVAR is updated
	\item \verb+Metadata (class.aggregatable)+ - Boolean value specifying if the PVAR is aggregatable or not. Aggregatability is determined based on the class to which a PVAR belongs. 
\end{itemize}
Apart from creating a Caliper attribute for each PVAR exported, two additional attributes are created for each \textit{watermark class} PVAR exported --- one that represents the number of times the watermark changes, and another that represents the cumulative change in the watermark PVAR.

\section{Querying and storing PVARs}
In the current design, all PVARs exported by the MPI library are queried when a snapshot is triggered. By integrating the \emph{MPIT} service along with the \emph{MPI} service, this would be useful in determining how various MPI function calls contribute to changes in PVAR values. Moreover, one can gather meaningful information by aggregating using MPI function names or annotated code regions as keys. Currently, we note about 10-15\% overheads in collecting PVARs during every snapshot event. This may not be a very scalable option, as this overhead would increase with a rise in number of PVARs exported.\\
Depending on the class of the PVAR, we either store the raw value read from the interface in the snapshot, or a derived metric.
\begin{itemize}
	\item \verb+MPI_T_PVAR_CLASS_TIMER+, \verb+MPI_T_PVAR_CLASS_AGGREGATE+, \\\verb+MPI_T_PVAR_CLASS_COUNTERS+: We store the difference between the current value and the "last value" for such PVARs in the snapshot. Storing and aggregating this derived value is more meaningful --- it helps us answer questions such as: How do different MPI functions contribute to this PVAR? Which MPI function is responsible for the highest value?
	\item \verb+MPI_T_PVAR_CLASS_STATE+, \verb+MPI_T_PVAR_CLASS_SIZE+: These PVARs are stored as is in the snapshot. Perhaps it may be more meaningful to view changes \emph{over time}, such as in a trace.
	\item \verb+MPI_T_PVAR_CLASS_HIGHWATERMARK+, \verb+MPI_T_PVAR_CLASS_LOWWATERMARK+: Along with storing the raw value for watermark PVARs, we store the derived metrics that represent the number of times the watermark changed, along with how much the watermark changed in the snapshot. By aggregating across MPI functions for example, we can answer questions such as: Which function most frequently pushed up / down a watermark? Which function was responsible for the highest cumulative change in a given watermark?
	\item \verb+MPI_T_PVAR_CLASS_LEVEL+, \verb+MPI_T_PVAR_CLASS_PERCENTAGE+: We store these PVARs as is in the snapshot. It maybe meaningful to view the average, maximum or minimum value for these PVARs, aggregated across MPI functions.
\end{itemize}

\section {Challenges in implementing MPI\_T in Caliper}
Use the LLNL slide info.

